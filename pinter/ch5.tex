\documentclass[12pt]{article}
\usepackage{amssymb,amsmath,latexsym}

\begin{document}
\title{A Book of Abstract Algebra: Solutions to Chapter 5}
\author{Tushar Tyagi}
\date{\today}
\maketitle

\section*{Notes}
A subgroup $S$ is called a subgroup of a group $G$, if:
\begin{enumerate}
  
\item It is closed on the given operation, i.e. the operation ($\cdot$) of two elements produces an element $\in$ $S$.
\item It is closed under inverse, i.e. the inverse of each element of $S$ is in $S$.
\end{enumerate}

Also, each subgroup is a group as well, and therefore follows the three group laws:
\begin{enumerate}
  \item Associativity
  \item Identity
  \item Inverse
\end{enumerate}

The $identity, e$ of the group is shared by the subgroup.

\subsubsection*{Trivial \& Proper Subgroups}
\begin{enumerate}
 \item The one-element subset $\{e\}$ and the entire group $G$ are the smallest and the largest subgroups of $G$ and are called \textit{trivial subgroups}.
 \item All the other subgroups of G are called \textit{proper subgroups}.
\end{enumerate}


\subsubsection*{Cyclic Groups and Subgroups}
If a group (or a subgroup) is generated by a single element, we call that group
\textit{Cyclic} and it is written as $\langle a \rangle $, where $a$ is called the \textit{generator} and is the single element which, along with the identity and $a^{-1}$, can define the entire group. 


\subsubsection*{Defining Equations}
A set of equations, involving only the generators and their inverses, is called a set of \textit{defining equations}. These equations can completely define the operation table of the group.


\section*{Solutions}
\label{sec:solutions}

\subsection*{Set A}

\begin{enumerate}
\item $G = \langle R, + \rangle, H = \{log a: a \in \mathbb{Q}, a > 0\}$
  \begin{itemize}
    \item 
      Addition:
      
      Let $a, b \in \mathbb{Q} \\
      log\ a + log\ b = log\ ab \\
      \because a, b \in \mathbb{Q},\\
      \therefore ab \in \mathbb{Q}, ab > 0, \\
      \Rightarrow log\ ab \in H$
    \item 
      Identity:
      
      The identity element would not change the value of $log\ a$ under addition. 
      $log\ 1$ or $0$ is the identity element, since:

      If $log\ a + log\ b = log\ a$, then $log\ b = 0$, and $b = 1$.


    \item 
      Inverse:
        \begin{alignat*}{3}
           &log\ a + log\ a^{-1} \ &= &\ e \\ 
          \Rightarrow & log\ a &= & -log\ a^{-1} \\
          \Rightarrow & log a   &= &\ log(\frac{1}{a^{-1}}) \\
          \Rightarrow & a      &= &\ \frac{1}{a^{-1}}
        \end{alignat*}
      
      Since $a \in \mathbb{Q}$, $\frac{1}{a^{-1}} \in \mathbb{Q}$, $\therefore log\ a^{-1} \in H$
       
    \end{itemize}

\item $G = \langle R, + \rangle, H = \{log a: a \in \mathbb{Z}, a > 0\}$
  \begin{itemize}
    \item 
      Addition: \\

      Same reasoning as previous question.
    \item 
      Inverse: \\
      As calculated in the previous question, $a^{-1} = \frac{1}{a}$
      Since $a \in \mathbb{Z}, a^{-1} \notin \mathbb{Z}$
  \end{itemize}


\item $G = \langle R, + \rangle, H = \{x \in \mathbb{R}: tan x \in \mathbb{Q} \}$

  \begin{itemize}
    \item 
      Addition: \\
      Let $x, y \in \mathbb{R}. \\
      \therefore tan(x+y) = \frac{tanx + tany}{1 - tanx \cdot tany}$

      If $x = y = 45^{\circ}$, then $tan\ x = tan\ y = 1$, which makes the denominator
      undefined, and therefore addition is not defined for $H$.

  \end{itemize}


\item $G = \langle R, \cdot \rangle, H = \{2^{n}3^{m}, m, n \in \mathbb{Z}\}$

  \begin{itemize}
    \item 
      Multiplication: \\
      Let $n, m, n', m' \in \mathbb{Z}. \\
      2^{n}3^{m} \cdot 2^{n'}3^{m'} = 2^{n+n'}3^{m+m'} \\
      \because n + n', m + m' \in \mathbb{Z} \\
      \therefore 2^{n+n'}3^{m+m'} \in H $

    \item 
      Inverse: \\
      Since $2^{n}3^{m} \in \mathbb{Z}, \\ $
      The inverse is $\frac{1}{2^{n}3^{m}} = 2^{-n}3^{-m}$
      $\because -n, -m \in \mathbb{Z}, \therefore  2^{-n}3^{-m} \in \mathbb{Z}$  

  \end{itemize}

\item $G = \langle R \times R, + \rangle, H = \{(x,y) : y = 2x \}$

  \begin{itemize}
    \item
      Addition: \\
      $(x,y) + (x',y') = (x+x, y+y') \\
      \because x,\ x',\ y,\ y' \in \mathbb{R}, \\
      \therefore x+x',\ y+y' \in \mathbb{R} \\
      \therefore (y+y') = 2(x+x')$

    \item 
      Inverse: \\
      $e = (0,0)$ 
      Inverse: $(-x, -y) \\
      \because y = 2x \\
      \Rightarrow -y = -2x \\
      \Rightarrow y' = 2x'$

  \end{itemize}

\item $G = \langle R \times  R, + \rangle, H = \{(x,y) : x^{2} + y^{2} > 0\}$

  \begin{itemize}
    \item 
      Addition \\
      $(x,y) + (x',y') = (x + x', y+y') \\
      \Rightarrow (x+x')^{2} + (y+y')^{2} > 0, \\
      \therefore$ Addition operation is defined for $H$.
      
    \item
      Inverse: \\
      $\because e = (0,0)$ \\
      Inverse is $(-x, -y)$ \\
      Let $x, y \in H, \\
      \therefore x^{2} + y^{2} > 0, \\
      \Rightarrow (-x)^{2} + (-y)^{2} > 0 \in H$ \\
      So inverse is defined.

  \end{itemize}

\item Let $C$ and $D$ be sets, with $C \subseteq D$. Prove that $P_C$ is a subgroup of $P_D$

  In a way, $G = \langle P_D , + \rangle, H = \langle P_c , + \rangle$
  
\begin{itemize}
  \item 
    Identity: \\ (common to both $P_C$ and $P_D$): $\{\phi\}$
  \item 
    Inverse: \\
    $A^{-1} = A$ \\
    We have already proved this in Chapter 3, Exercise C.
  \item 
    Addition: \\
    Let $A, B \subseteq H$, \\
    $A + B = (A - B) \cup (B - A)$
    Since $A \subseteq P_C, (A - B) \subseteq P_C$. Similarly, $(B - A) \subseteq P_C$. So the operation of symmetric difference is closed on subgroup $H$. 

\end{itemize}

  \end{enumerate}


\end{document}
